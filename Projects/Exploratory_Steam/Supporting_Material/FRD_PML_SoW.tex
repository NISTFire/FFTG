\documentclass[12pt,letterpaper]{article}
\usepackage{graphicx}
\usepackage{tikz}
\usepackage{amsmath}
\usepackage{pdfsync}
\usepackage{url}
\usepackage{hyperref}
\usepackage{microtype}
\usepackage{siunitx}
\usepackage{tabularx,ragged2e,booktabs,caption}
\usepackage{caption}
\frenchspacing
\hypersetup{colorlinks=true, urlcolor=blue, citecolor=black, linkcolor=black}
\urlstyle{same}

\renewcommand{\d}{\mathrm{d}}

\title{Assessment of Fire Fighter Steam Burn Potential by Quantifying Water Vapor Transfer in Protective Ensemble}
\date{\today}

\begin{document}

\maketitle

\section{Objective}
\label{sec:objective}

To understand the potential for steam burns by quantifying the dynamic transfer of water vapor through layers of Fire Fighter Protective Clothing (FFPC).

\section{Problem}
\label{sec:problem}

Structural fire fighters work in hot and wet environment.  Water vapor, commonly addressed as steam, comes from three sources, 1) steam/water vapor generated by the fire outside the FFPC, 2) steam/water vapor generated during fire suppression outside the FFPC, and 3) perspiration inside the FFPC.

The water vapor that diffuses across the fabric layers has significant amount of heat energy that could be released if this moisture condenses on the skin of the wearer.  Condensation of water vapor/steam releases the latent heat of vaporization that has potential to produce skin burns, often referred to as steam burns.  Currently, very little is known about water vapor transfer from external fire environment through fire fighters protective clothing.

\section{New Technical Idea}
\label{sec:tech_idea}

The new technical idea is to examine a bench-scale test device for measuring water vapor transfer in protective clothing that is related to occurrence of steam burns.  The humidity generator along with the steam chamber available in Sensor Science Division (SSD), PML will be used to create humid environment that is typically found during fire suppression activities.  The water vapor that diffuses through the clothing system will be determined by measuring the rate of increase in relative humidity (RH) and temperature for a given period of time, maximum increase in relative humidity, and time it takes for the RH to reach a steady state level.  Based on these parameters, we can derive a moisture transfer index which can indicate water vapor modulating capabilities for a given fabric assembly.  

\section{Impact}

Knowledge gained from the experiments performed using the newly developed test device will improve general understanding of potential for burns from external hot water vapor/steam exposure with FFPC meeting current NFPA standards.  

\section{Statement of Work}

The (SSD) will conduct exposures to firefighter protective clothing samples in an instrumented mount (provided by FRD) in their environmental chamber.  The exposure conditions will be developed between the technical leads, Christopher Meyer (SSD) and Craig Weinschenk (FRD).   SSD will collect data from the environmental chamber that documents the conditions within the test area of the chamber.  FRD will provide instrumentation and collect the data from the samples.  The exposure test will be completed by January 31, 2015.  

\begin{table}
\centering
\captionof{table}{Exposure Tests}\label{tab:tests}
\begin{tabular}{lllll}
\toprule[1.5pt]
Test \#       &  Material                   &  Duration       &  Environment       &  Reps  \\
\midrule
1               &  None                       &  Steady-State   &  85~$^{\circ}$C \& 95~\%RH   &  5  \\[.25cm]
2               &  Full PPE (3 layer)         &  10~min$^*$     &  85~$^{\circ}$C \& 95~\%RH   &  5  \\[.25cm]
3               &  Woven Fabric (hood)        &  10~min$^*$     &  85~$^{\circ}$C \& 95~\%RH   &  5  \\[.25cm]
4               &  None                       &  Steady-State   &  85~$^{\circ}$C \& 65~\%RH    &  5 \\[.25cm]
5               &  Full PPE (3 layer)         &  10~min$^*$     &  85~$^{\circ}$C \& 65~\%RH    &  5  \\[.25cm]
6               &  Woven Fabric (hood)        &  10~min$^*$     &  85~$^{\circ}$C \& 65~\%RH    &  5 \\[.25cm]
7$^+$           &  None                       &  Steady-State   &  85~$^{\circ}$C \& 35~\%RH    &  5 \\[.25cm]
8$^+$           &  Full PPE (3 layer)         &  10~min$^*$     &  85~$^{\circ}$C \& 35~\%RH    &  5 \\[.25cm]
9$^+$           &  Woven Fabric (hood)        &  10~min$^*$     &  85~$^{\circ}$C \& 35~\%RH   &  5 \\[.25cm]
\bottomrule[1.25pt]
\end{tabular}
\footnotesize
\raggedright
\\ $^*$ 10 min for first replicate. Adjust as necessary.
\\ $^+$ If necessary
\normalsize
\end{table}

\clearpage

\section{Deliverables for PML}
\begin{itemize}
\item Exposure experiments completed by January 31, 2015.
\item Written report describing the test apparatus, arrangement and exposure protocol and conditions. Report delivered to Dr. Weinschenk by February 24, 2015.
\end{itemize}

\section{Technical Point of Contacts:}

SSD: Christopher Meyer \href{mailto:christopher.meyer@nist.gov}{christopher.meyer@nist.gov} EXT: 4825 \\
FRD: Craig Weinschenk \href{matilto:craig.weinschenk@nist.gov}{craig.weinschenk@nist.gov} EXT: 6899  

\end{document}
