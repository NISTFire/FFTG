\documentclass{article}
\usepackage[utf8]{inputenc}

\title{Spartanburg 2013 Test Descriptions}
\author{Keith Stakes}


\begin{document}

\maketitle

\section{214 Folsom Street}

The first test house was a single story, single family residential structure of type 5 construction addressed as 214 Folsom Street. The non-burn rooms had a limited-combustible finish comprised of either paper-covered gypsum board or a finished plaster board.  The burn rooms had a combustible finish comprised of OSB sheets for wall linings.  Two tests were conducted in this structure: 1) Living room fire with exterior attack and 2) Bedroom fire with exterior attack.

\subsection{Test 1: Living Room Fire}

Ignition begins on a modern-style sofa in the living room on the A/D corner of the structure with all doors and windows closed. The fire grew uninhibited and reach the vent-limited stage. Fire fighting crew ventilated the living room window on the D side of the structure.The fire was allowed to stabilize with the new ventilation opening. The fire fighting crew then applied water, via a straight stream flow, from the exterior into the D side window for suppression.  After the exterior suppression, the front living room window on the A side of the structure was ventilated.  Water was then applied from the exterior on the A side into the living room window for further suppression. The fire fighting crew then opened the front door and made entry to complete suppression.

\subsection{Test 2: Bedroom Fire}

The fire was ignited on a modern-style soda in the bedroom located near the B/C corner of the structure.  The living room used during the previous burn was isolated from the remaining portion of the building for this test to not alter the results. The fire was allowed to grow uninhibited and reach the vent-limited stage. The fire self-vented the upper portion of the bedroom window on the B side of the structure.  The fire fighting crew ventilated a window on the D side of the building in the adjacent room located near the C/D corner.  The fire then self-vented the remaining portion of the window in the room of origin.  The small window into the kitchen on the B side of the structure was then ventilated.  Shortly thereafter, the fire fighting crew made an exterior attack via a narrow fog into the bedroom from the self-vented B side window.

\section{215 Folsom Street}

The second test house was a single story, single family residential structure of type 5 construction addressed as 215 Folsom Street. The non-burn rooms had a limited-combustible finish comprised of either paper-covered gypsum board or a finished plaster board.  The burn rooms had a combustible finish comprised of OSB sheets for wall linings.  Two tests were conducted in this structure: 1) Living room fire with exterior attack and 2) Bedroom fire with exterior attack.  

\subsection{Test 1: Living Room Fire}

The fire was ignited in the front living room located near the A/B corner of the structure on a modern-style sofa. The fire grew uninhibited until reaching the ventilation-limited stage. While in the growth stage, the fire self-vented the living room window on the A side of the structure, near the A/B corner. After the fire fighting unit officer completed a 360 degree size-up of the structure, the front door was opened from the exterior.  The living room window on the B side of the structure then self-vented as well.  Shortly thereafter, the pressure from the fire compartment closed the front door.  After a short period of time, the fire fighting crew re-opened the front door to the building. The fire fighting crew began an attack on the structure via a single smooth bore hose-line first knocking down the fire on the porch.  The nozzle was then directed into the A side living room window and front doorway. The fire fighting crew then completed suppression and overhauled the porch area to complete mop-up procedures.

\subsection{Test 2: Bedroom Fire}

The fire was ignited in a rear bedroom near the C/D corner of the structure on a modern-style sofa.  The C side window to the bedroom was half open.  The C side window in the adjoining room also had a window in the half open position. The living room used during the previous burn was isolated from the remaining portion of the building for this test to not alter the results. The fire was allowed to grow uninhibited and reach the vent-limited stage. The fire then self-vented a portion of the bedroom window on the D side. Once the temperatures leveled off in the fire room, the fire fighting crew ventilated the remaining portion of the window in the adjoining room, also on the D side of the structure. As the fire continued to grow, the fire fighting crew ventilated the remaining portion of the bedroom window in the room of origin. The fire room reached flashover conditions.  The fire fighting crew then applied water into the room of origin via a single hose-line with a narrow fog pattern until the fire was completely suppressed.     

\section{218 Folsom Street}

The third test house was a single story, single family residential structure of type 5 construction addressed as 218 Folsom Street. The non-burn rooms had a limited-combustible finish comprised of either paper-covered gypsum board or a finished plaster board.  The burn rooms had a combustible finish comprised of OSB sheets for wall linings.  Two tests were conducted in this structure: 1) Bedroom fire with exterior attack and 2) Bedroom fire with exterior attack. 

\subsection{Test 1: Bedroom Fire}

The fire was ignited in a rear bedroom near the C/D corner of the structure on a modern-style mattress with foam padding atop. All doors and windows to the exterior of the structure were closed at the beginning of the test. The fire grew uninhibited until reaching the vent-limited stage. The officer of the fire fighting crew completed a 360 degree size-up of the structure to assess conditions.  Shortly thereafter, the fire fighting crew ventilated the bedroom window on the D side of the building. The fire was allowed to stabilize given the new ventilation opening. The fire fighting crew then made an attack on the fire by flowing a single hand-line with a smooth bore nozzle into the room of origin through the D side window from the exterior.

\subsection{Test 2: Bedroom Fire}

Ignition occurred in a rear bedroom near the B/C corner of the structure on a modern-style couch. All doors and windows to the exterior of the structure were closed at the beginning of the test. The bedroom used during the previous burn was isolated from the remaining portion of the building for this test to not alter the results. The fire grew uninhibited until reaching the vent-limited stage. The officer of the fire fighting crew completed a 360 degree size-up and then the fire fighting crew ventilated the target room window on the B side of the building. This was the window to the adjoining room in the structure located near the A/B corner. The fire was allowed to stabilize given the new ventilation opening. The fire fighting crew then ventilated the bedroom window to the room of origin on the B side next.  This was followed up by the fire fighting crew making an attack on the fire by flowing a single hand-line with a smooth bore nozzle into the room of origin through the B side window from the exterior.

\section{219 Folsom Street}

The third test house was a single story, single family residential structure of type 5 construction addressed as 219 Folsom Street. The non-burn rooms had a limited-combustible finish comprised of either paper-covered gypsum board or a finished plaster board.  The burn rooms had a combustible finish comprised of OSB sheets for wall linings.  Two tests were conducted in this structure: 1) Bedroom fire with exterior attack and 2) Living Room fire with exterior attack.    

\subsection{Test 1: Bedroom Fire}

Ignition occurred in a rear bedroom near the B/C corner of the structure on a modern-style mattress with foam padding. All doors and windows to the exterior of the structure were closed at the beginning of the test. The fire grew uninhibited until reaching the vent-limited stage.  The front door to the structure was then opened by the fire fighting crew to simulate a crew making entry or a homeowner leaving the door open upon evacuation. The fire continued to grow uninhibited until the fire room reached flashover conditions. The fire fighting crew then ventilated the bedroom window on the B side of the structure in the room of origin. The fire was then allowed to stabilize given the new ventilation opening. An unmanned monitor nozzle connected to a single hose-line was placed inside the structure at the entry to the fire room.  This was remotely activated to simulate a fire fighting crew making an interior attack on the seat of the fire.  The water was flowed until the fire was suppressed. The fire fighting crew then completed the test by making entry to the structure to complete overhaul and mop-up procedures. 

\subsection{Test 2: Living Room Fire}
  
The fire was ignited in the front living room located near the A/D corner of the structure on a modern-style sofa. The bedroom used during the previous burn was isolated from the remaining portion of the building for this test to not alter the results. The window to the living room on the D side of the structure was half-opened at the start of the test. The fire grew uninhibited until reaching the ventilation-limited stage. The window on the D side in the adjacent room near the C/D corner of the structure was ventilated next.  The upper portion of the A side living room window self-ventilated due to the fire.  Shortly thereafter, the fire fighting crew ventilated the remaining portion of the D side living room window. Once the fire had self-ventilated the remaining portion of the A side living room window, it was allowed to stabilize.  The fire fighting crew then conducted an exterior attack via the D side living room window with a single hose-line with a smooth bore nozzle. 

\section{225 Folsom Street}

 The third test house was a single story, single family residential structure of type 5 construction addressed as 225 Folsom Street. The non-burn rooms had a limited-combustible finish comprised of either paper-covered gypsum board or a finished plaster board.  The burn rooms had a combustible finish comprised of OSB sheets for wall linings.  Two tests were conducted in this structure: 1) Bedroom fire with exterior attack and 2) Bedroom fire with exterior attack. 

\end{document}
